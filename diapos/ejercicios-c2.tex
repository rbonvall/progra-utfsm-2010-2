\documentclass[10pt]{beamer}
\usepackage[spanish]{babel}
\usepackage[utf8]{inputenc}
\usepackage{xcolor}
\usepackage{listings}
\usepackage{palatino}
\usepackage{fancyvrb}
\usepackage{amsmath}
\usepackage{pifont}

\newcommand{\T}{\ding{51}}
\newcommand{\F}{}

\usecolortheme{crane}
\usefonttheme{serif}

\title{Ejercicios para el certamen 2}
\author{
  Roberto Bonvallet \\
  \url{roberto.bonvallet@usm.cl} \\
  \url{http://progra.8o.cl}
}

\lstloadlanguages{fortran}
\lstset{language=fortran,basicstyle=\small,%
        morestring=[b]',stringstyle=\it,showstringspaces=false}
\fvset{formatcom=\small,frame=single,gobble=6,commandchars=\\\{\}}

\begin{document}
  \begin{frame}
    \maketitle
  \end{frame}

  \begin{frame}[fragile]
    Escriba un programa
    que cuente cuántas veces aparece cada letra
    en una oración ingresada por el usuario.
    \begin{Verbatim}
       Ingrese una oracion
      \alert{Puro Chile es tu cielo azulado}
       a           2
       c           2
       d           1
       e           3
       h           1
       i           2
       l           3
       o           3
       p           1
       r           1
       s           1
       t           1
       u           3
       z           1
    \end{Verbatim}

\end{frame}

  \begin{frame}
    \lstinputlisting[linerange=1-16]{../programas/cuenta-letras.f95}
    ...
  \end{frame}

  \begin{frame}[fragile]
    \frametitle{ASCII code}
    \begin{verbatim}
           3-  4-  5-  6-  7-  8-  9- 10- 11- 12-
         ----------------------------------------
      -0 |     (   2   <   F   P   Z   d   n   x
      -1 |     )   3   =   G   Q   [   e   o   y
      -2 |     *   4   >   H   R   \   f   p   z
      -3 | !   +   5   ?   I   S   ]   g   q   {
      -4 | "   ,   6   @   J   T   ^   h   r   |
      -5 | #   -   7   A   K   U   _   i   s   }
      -6 | $   .   8   B   L   V   `   j   t   ~
      -7 | %   /   9   C   M   W   a   k   u  DEL
      -8 | &   0   :   D   N   X   b   l   v
      -9 |     1   ;   E   O   Y   c   m   w
    \end{verbatim}

    \begin{itemize}
      \item \lstinline+iachar('m')+ $\longrightarrow 109$
      \item \lstinline+achar(78)+ $\longrightarrow$ \lstinline+'N'+
    \end{itemize}

\end{frame}

  \begin{frame}
    \frametitle{Ejercicio}
    La asistencia de los alumnos a clases
    puede ser llevada en una tabla como la siguiente:

    \begin{tabular}{|l|c|c|c|c|c|c|c|}\hline
      Clase    & 1& 2& 3& 4& 5& 6& 7\\\hline\hline
      Pepito   &\T&\T&\T&\F&\F&\F&\F\\\hline
      Yayita   &\T&\T&\T&\F&\T&\F&\T\\\hline
      Fulanita &\T&\T&\T&\T&\T&\T&\T\\\hline
      Panchito &\T&\T&\T&\F&\T&\T&\T\\\hline
    \end{tabular}
    \vspace{4ex}

    Escriba un programa que permita ingresar
    los nombres de los alumnos y
    sus asistencias a cada una de las clases,
    y que indique cuál fue el alumno
    que asistió a más clases.
  \end{frame}

  \begin{frame}[fragile]
    \frametitle{Ejercicio}
    \begin{Verbatim}
       Ingrese los nombres de los alumnos
      \alert{Pepito}
      \alert{Yayita}
      \alert{Fulanita}
      \alert{Panchito}
       Ingrese asistencia de Pepito
      \alert{T T T F F F F}
       Ingrese asistencia de Yayita
      \alert{T T T F T F T}
       Ingrese asistencia de Fulanita
      \alert{T T T T T T T}
       Ingrese asistencia de Panchito
      \alert{T T T F T T T }
       El alumno mas responsable fue Fulanita
       Fulanita asistió a 7 clases
    \end{Verbatim}

\end{frame}

  \begin{frame}
    \frametitle{Ejercicio}
    \lstinputlisting[linerange=1-20]{../programas/asistencia.f95}
    ...
  \end{frame}

  \begin{frame}[fragile]
    \frametitle{Control 4}
    \pause
    Complete el siguiente programa
    escribiendo el subprograma correspondiente:
    \begin{lstlisting}[gobble=6,frame=single]
      program pitagoras
          implicit none
          real :: a, b

          print *, 'Ingrese cateto 1:'
          read *, a
          print *, 'Ingrese cateto 2:'
          read *, b
          print *, 'La hipotenusa es ', hipotenusa(a, b)

      contains

          ...

      end program pitagoras
    \end{lstlisting}
    
\end{frame}

\end{document}
