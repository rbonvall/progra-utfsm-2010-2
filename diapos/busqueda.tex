\documentclass[10pt]{beamer}
\usepackage[spanish]{babel}
\usepackage[utf8]{inputenc}
\usepackage{xcolor}
\usepackage{listings}
\usepackage{palatino}
\usepackage{fancyvrb}

\usecolortheme{crane}
\usefonttheme{serif}

\title{Búsqueda en arreglos}
\author{
  Roberto Bonvallet \\
  \url{roberto.bonvallet@usm.cl} \\
  \url{http://progra.8o.cl}
}

\lstloadlanguages{fortran}
\lstset{language=fortran,basicstyle=\small,%
        morestring=[b]',stringstyle=\it,showstringspaces=false}
\fvset{formatcom=\small,frame=single,gobble=6,commandchars=\\\{\}}

\begin{document}
  \begin{frame}
    \maketitle
  \end{frame}

  \begin{frame}{Ejercicios de hoy}
    \begin{itemize}
      \item Búsqueda lineal
      \item Búsqueda binaria
      \item Control 3
    \end{itemize}
  \end{frame}

  \begin{frame}[fragile]
    Determinar si un valor está o no en un arreglo.

    \begin{lstlisting}[gobble=6]
      integer, dimension(N) :: a = &
          (/65, 27,  7, 21, 59, -9, 25, 24, 59, 57, &
            43, 53,  0, 50,  3, 30, 29, 15, 20, 31, &
            24, 47, -1,  1, 44, -6, 24, 14, 26, 41, &
            73, 35, 65, 45, 36, 66, 19, 17, 15, 38/)
    \end{lstlisting}

    \begin{Verbatim}
       ¿Qué valor quiere buscar?
      \alert{48}
       48 no está en el arreglo.
    \end{Verbatim}

    \begin{Verbatim}
       ¿Qué valor quiere buscar?
      \alert{29}
       29 sí está en el arreglo.
    \end{Verbatim}

\end{frame}

  \begin{frame}[fragile]
    Determinar en qué posición del arreglo está un valor.

    \begin{lstlisting}[gobble=6]
      integer, dimension(N) :: a = &
          (/65, 27,  7, 21, 59, -9, 25, 24, 59, 57, &
            43, 53,  0, 50,  3, 30, 29, 15, 20, 31, &
            24, 47, -1,  1, 44, -6, 24, 14, 26, 41, &
            73, 35, 65, 45, 36, 66, 19, 17, 15, 38/)
    \end{lstlisting}

    \begin{Verbatim}
       ¿Qué valor quiere buscar?
      \alert{48}
       48 no está en el arreglo.
    \end{Verbatim}

    \begin{Verbatim}
       ¿Qué valor quiere buscar?
      \alert{29}
       29 está en la posición 17
    \end{Verbatim}

\end{frame}

  \begin{frame}[fragile]
    Determinar en qué posición del arreglo
    hay un múltiplo de 19.

    \begin{lstlisting}[gobble=6]
      integer, dimension(N) :: a = &
          (/65, 27,  7, 21, 59, -9, 25, 24, 59, 57, &
            43, 53,  0, 50,  3, 30, 29, 15, 20, 31, &
            24, 47, -1,  1, 44, -6, 24, 14, 26, 41, &
            73, 35, 65, 45, 36, 66, 19, 17, 15, 38/)
    \end{lstlisting}

    \begin{Verbatim}
       El múltiplo de 19 está en la posición 9.
    \end{Verbatim}

\end{frame}

  \begin{frame}[fragile]
    Determinar en qué posición de un arreglo \alert{ordenado}
    está un valor.
    
    \begin{lstlisting}[gobble=6]
      integer, dimension(40) :: a = &
          (/-9, -6, -1,  0,  1,  3,  7, 14, 15, 15, &
            17, 19, 20, 21, 24, 24, 24, 25, 26, 27, &
            29, 30, 31, 35, 36, 38, 41, 43, 44, 45, &
            47, 50, 53, 57, 59, 59, 65, 65, 66, 73/)
    \end{lstlisting}

    \begin{Verbatim}
       ¿Qué valor quiere buscar?
      \alert{48}
       48 no está en el arreglo.
    \end{Verbatim}

    \begin{Verbatim}
       ¿Qué valor quiere buscar?
      \alert{30}
       30 está en la posición 22.
    \end{Verbatim}

\end{frame}

  \begin{frame}
    \begin{columns}
      \column{.35\textwidth}
      \tiny
      \begin{tabular}{r|r|l}\cline{2-2}
         1 & \only<-2>{--9} & \\
         2 & \only<-2>{--6} & \\
         3 & \only<-2>{--1} & \\
         4 & \only<-2>{  0} & \\
         5 & \only<-2>{  1} & \\
         6 & \only<-2>{  3} & \\
         7 & \only<-2>{  7} & \\
         8 & \only<-2>{ 14} & \\
         9 & \only<-2>{ 15} & \\
        10 & \only<-2>{ 15} & \\
        11 & \only<-2>{ 17} & \\
        12 & \only<-2>{ 19} & \\
        13 & \only<-2>{ 20} & \\
        14 & \only<-2>{ 21} & \\
        15 & \only<-2>{ 24} & \\
        16 & \only<-2>{ 24} & \\
        17 & \only<-2>{ 24} & \\
        18 & \only<-2>{ 25} & \\
        19 & \only<-2>{ 26} & \\
        20 & \only<-2>{ 27} & \only<2>{$\longleftarrow$ ¿\texttt{a(20) == 30}?} \\
        21 &          { 29} & \\
        22 &  \alert<9>{30} & \only<8-9>{$\longleftarrow$ ¿\texttt{a(22) == 30}?} \\
        23 &          { 31} & \\
        24 &          { 35} & \\
        25 & \only<-6>{ 36} & \only<6>{$\longleftarrow$ ¿\texttt{a(25) == 30}?} \\
        26 & \only<-6>{ 38} & \\
        27 & \only<-6>{ 41} & \\
        28 & \only<-6>{ 43} & \\
        29 & \only<-6>{ 44} & \\
        30 & \only<-4>{ 45} & \only<4>{$\longleftarrow$ ¿\texttt{a(30) == 30}?} \\
        31 & \only<-4>{ 47} & \\
        32 & \only<-4>{ 50} & \\
        33 & \only<-4>{ 53} & \\
        34 & \only<-4>{ 57} & \\
        35 & \only<-4>{ 59} & \\
        36 & \only<-4>{ 59} & \\
        37 & \only<-4>{ 65} & \\
        38 & \only<-4>{ 65} & \\
        39 & \only<-4>{ 66} & \\
        40 & \only<-4>{ 73} & \\
        \cline{2-2}
      \end{tabular}
      \column{.65\textwidth}
        Buscar el valor $x = 30$ en el arreglo $a$:
        \begin{itemize}
          \item<1-> Buscar $x$ desde $a(1)$ hasta $a(40)$
          \item<2-> Comparar con $a(20)$\dots\quad \only<3->{$x > a(20)$}
          \item<3-> Buscar $x$ desde $a(21)$ hasta $a(40)$
          \item<4-> Comparar con $a(30)$\dots\quad \only<5->{$x < a(30)$}
          \item<5-> Buscar $x$ desde $a(21)$ hasta $a(29)$
          \item<6-> Comparar con $a(25)$\dots\quad \only<7->{$x < a(25)$}
          \item<7-> Buscar $x$ desde $a(21)$ hasta $a(24)$
          \item<8-> Comparar con $a(22)$\dots\quad \only<9->{$x == a(22)$}
          \item<9-> $x$ está en la posición 22
        \end{itemize}
    \end{columns}
  \end{frame}

  \begin{frame}
    ¿Cuánto se demora la búsqueda en el peor caso?

    \begin{tabular}{l|l|l}
      Tamaño del arreglo & Lineal\hfil & Binaria \\ \hline
      16            & \uncover<2->{16}            & \uncover<3->{4}  \\
      1000          & \uncover<2->{1000}          & \uncover<3->{10} \\
      1 millón      & \uncover<2->{1 millón}      & \uncover<3->{20} \\
      1000 millones & \uncover<2->{1000 millones} & \uncover<3->{30} \\
      $N$           & \uncover<4>{$N$} & \uncover<4>{$\log_2 N$} \\
    \end{tabular}
  \end{frame}

  \begin{frame}{Control 3}
    \only<2>{
      Escriba un programa que:
      \begin{itemize}
        \item pida al usuario ingresar 1000 números enteros, y
        \item muestre por pantalla
          todos los números ingresados
          que sean pares y menores que el último.
      \end{itemize}
    }
    
  \end{frame}

\end{document}

