\documentclass[10pt]{beamer}
\usepackage[spanish]{babel}
\usepackage[utf8]{inputenc}
\usepackage{xcolor}
\usepackage{listings}
\usepackage{palatino}
\usepackage{fancyvrb}
\usepackage{amsmath}

\usecolortheme{crane}
\usefonttheme{serif}

\title{Paso de parámetros}
\author{
  Roberto Bonvallet \\
  \url{roberto.bonvallet@usm.cl} \\
  \url{http://progra.8o.cl}
}

\lstloadlanguages{fortran}
\lstset{language=fortran,basicstyle=\small,%
        morestring=[b]',stringstyle=\it,showstringspaces=false}
\fvset{formatcom=\small,frame=single,gobble=6,commandchars=\\\{\}}

\begin{document}
  \begin{frame}
    \maketitle
  \end{frame}

  \begin{frame}[fragile]
    \frametitle{Subprogramas}
    \begin{lstlisting}[frame=single,gobble=6,basicstyle=\large]
      program nombre

          ! declaraciones

          ! programa

      contains

          ! subprogramas

      end program nombre
    \end{lstlisting}
    
\end{frame}

  \begin{frame}[fragile]
    \frametitle{Subprogramas}
    \begin{columns}
      \column{.5\textwidth}
        \begin{lstlisting}[frame=single,gobble=10]
          function fact(n) result(f)
              integer :: n, f, i

              f = 1
              do i = 1, n
                  f = f * i
              end do
          end function factorial
        \end{lstlisting}
      \column{.5\textwidth}
        \begin{lstlisting}[frame=single,gobble=10]
          subroutine llenar(n)
              integer :: n, i

              do i = 1, n
                  read *, a(i)
              end do

          end subroutine llenar
        \end{lstlisting}
    \end{columns}

\end{frame}

  \begin{frame}[fragile]
    \frametitle{¿Qué pasa al modificar el parámetro?}
    \begin{columns}
      \column{0.5\textwidth}
        \lstinputlisting[frame=single]{../programas/pp-no-1.f95}
      \column{0.5\textwidth}
        \lstinputlisting[frame=single]{../programas/pp-no-2.f95}
    \end{columns}
    \pause
    \begin{columns}
      \column{0.5\textwidth}
        \begin{Verbatim}
          3
        \end{Verbatim}
      \column{0.5\textwidth}
        \begin{Verbatim}
          2
        \end{Verbatim}
    \end{columns}

\end{frame}

  \begin{frame}
    A veces queremos que los parámetros cambien\dots:
    \lstinputlisting[frame=single]{../programas/intercambio.f95}
  \end{frame}

  \begin{frame}
    \dots y otras veces queremos que los parámetros no cambien:
    \lstinputlisting[frame=single]{../programas/factorial-fn-in.f95}
  \end{frame}

  \begin{frame}
    \begin{itemize}
      \item \lstinline+intent(in)+:
      \begin{itemize}
        \item el compilador no permite que el parámetro sea modificado;
        \item se puede pasar el valor de cualquier expresión
          como parámetro del subprograma.
      \end{itemize}
      \item \lstinline+intent(in out)+:
      \begin{itemize}
        \item sí se puede modificar el valor del parámetro;
        \item sólo se puede pasar una variable como parámetro.
      \end{itemize}
      \item sin \lstinline+intent+:
      \begin{itemize}
        \item todo está permitido;
        \item para evitar confusión, mejor siempre declarar \lstinline+intent+.
      \end{itemize}
    \end{itemize}
  \end{frame}

  \begin{frame}
    \frametitle{Ejercicio: ruteo}
    \lstinputlisting{../programas/comb-con-intent.f95}
  \end{frame}

  \begin{frame}
    \frametitle{Paso de arreglos como parámetros}
    \lstinputlisting{../programas/paso-arreglos.f95}
  \end{frame}

  \begin{frame}[fragile]{Ejercicio}
    Escriba un programa que pida cinco números
    e indique cuál es la mediana.

    \begin{Verbatim}
       Ingrese 5 números
      \alert{3.1}
      \alert{2.0}
      \alert{5.4}
      \alert{1.7}
      \alert{4.3}
       La mediana es 3.100000
    \end{Verbatim}

\end{frame}

\end{document}
