\documentclass[12pt]{beamer}
\usepackage[spanish]{babel}
\usepackage[utf8]{inputenc}
\usepackage{xcolor}
\usepackage{listings}
\usepackage{palatino}
\usepackage{fancyvrb}
\usepackage{amsmath}
\usepackage{tikz}

\usecolortheme{crane}
\usefonttheme{serif}

\title{Registros}
\author{
  Roberto Bonvallet \\
  \url{roberto.bonvallet@usm.cl} \\
  \url{http://progra.8o.cl}
}

\lstloadlanguages{fortran}
\lstset{language=fortran,basicstyle=\small,%
        morestring=[b]',stringstyle=\it,showstringspaces=false}
\fvset{formatcom=\small,frame=single,gobble=6,commandchars=\\\{\}}

\begin{document}
  \begin{frame}
    \maketitle
  \end{frame}

  \begin{frame}
    \frametitle{Registros}

    \begin{tikzpicture}[scale=0.4]
      \draw (0, 1) rectangle (28, 6);
      \draw (1, 3) rectangle +(8, 2);
      \node at (5, 4) {20};
      \node at (5, 2) {\lstinline!integer :: dia!};
      \draw (10, 3) rectangle +(8, 2);
      \node at (14, 4) {10};
      \node at (14, 2) {\lstinline!integer :: mes!};
      \draw (19, 3) rectangle +(8, 2);
      \node at (23, 4) {2010};
      \node at (23, 2) {\lstinline!integer :: anno!};
      \node at (14, 0) {\lstinline!type(fecha) :: hoy!};
    \end{tikzpicture}
    \pause
    \vspace{2em}
    \lstinputlisting[linerange=4-8]{../programas/registro-fecha.f95}

  \end{frame}

  \begin{frame}[fragile]
    \frametitle{Asignación de campos}
    \begin{lstlisting}
      type :: fecha
          integer :: dia
          integer :: mes
          integer :: anno
      end type fecha

      type(fecha) :: hoy
    \end{lstlisting}
    \pause
    \begin{lstlisting}
      hoy% dia = 20
      hoy% mes = 10
      hoy% anno = 2010
\end{lstlisting}
\pause
    \begin{lstlisting}
      hoy = fecha(20, 10, 2010)
    \end{lstlisting}


\end{frame}

  \begin{frame}
    \frametitle{Registro más complicado}
    \lstinputlisting[linerange=10-17]{../programas/registro-alumno.f95}
  \end{frame}

  \begin{frame}[fragile]
    \frametitle{Registro más complicado}
    \begin{lstlisting}
      a% nombre = 'Perico'
      a% apellido = 'Los Palotes'
      a% nacimiento% dia = 4
      a% nacimiento% mes = 12
      a% nacimiento% anno = 1990
      a% notas(1) = 97
      a% notas(2) = 20
      a% notas(3) = 55
    \end{lstlisting}
    \pause
    \begin{lstlisting}
      b = alumno('Fulanita', 'De Tal',   &
                 fecha(12, 10, 1991),    &
                 (/50, 12, 95/))
    \end{lstlisting}

\end{frame}

  \begin{frame}[fragile]
    \frametitle{Arreglo de registros}
    \begin{lstlisting}[gobble=6]
      type(alumno), dimension(N) :: alumnos
    \end{lstlisting}
    \vspace{1em}
    \begin{tabular}{|l||l||r|r|r||r|r|r|}\hline
      Perico & Los Palotes &  4 & 12 & 1990 & 97 & 20 & 55 \\\hline
      Fulanita & De Tal    & 12 & 10 & 1991 & 50 & 12 & 95 \\\hline
      Yayita & Vinagre     &  3 & 11 & 1990 & 80 & 91 & 84 \\\hline
      John & Doe           & 28 & 12 & 1989 & 100 & 4 & 50 \\\hline
    \end{tabular}
    \vspace{1em}
    \pause
    \begin{itemize}
      \item \lstinline!alumnos(3) % nombre! \pause
      \item \lstinline!alumnos(2) % nacimiento % anno! \pause
      \item \lstinline!alumnos(4) % notas(1)! \pause
      \item \lstinline!alumnos(:) % apellido! \pause
      \item \lstinline!sum(alumnos(1) % notas) / 3.0! \pause
      \item \lstinline!sum(alumnos(:) % notas(1)) / 4.0!
    \end{itemize}

\end{frame}

  \begin{frame}[fragile]
    \frametitle{Llenar tabla}
    \begin{lstlisting}[gobble=6]
      print *, 'Ingrese datos alumnos:'
      do i = 1, N
          read *, alumnos(i)
      end do
    \end{lstlisting}
    \begin{Verbatim}
       Ingrese datos alumnos:
      \alert{Perico LosPalotes  4 12 1990  97 20 55}
      \alert{Fulanita DeTal  12 10 1991  50 12 95}
      \alert{Yayita Vinagre  3 11 1990  80 91 84}
      \alert{John Doe  28 12 1989  100 4 50}
    \end{Verbatim}

\end{frame}

\end{document}
