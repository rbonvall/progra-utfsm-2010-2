\documentclass[10pt]{beamer}
\usepackage[spanish]{babel}
\usepackage[utf8]{inputenc}
\usepackage{xcolor}
\usepackage{listings}
\usepackage{palatino}
\usepackage{fancyvrb}

\usecolortheme{crane}
\usefonttheme{serif}

\title{Ejercicios con arreglos}
\author{
  Roberto Bonvallet \\
  \url{roberto.bonvallet@usm.cl} \\
  \url{http://progra.8o.cl}
}

\lstloadlanguages{fortran}
\lstset{language=fortran,basicstyle=\small,%
        morestring=[b]',stringstyle=\it,showstringspaces=false}
\fvset{formatcom=\small,frame=single,gobble=6,commandchars=\\\{\}}

\begin{document}
  \begin{frame}
    \maketitle
  \end{frame}

  \begin{frame}[fragile]
    Ordenar los valores ingresados:

    \begin{Verbatim}
       Ingrese 10 datos:
      \alert{3}
      \alert{1}
      \alert{5}
      \alert{8}
      \alert{4}
      \alert{7}
      \alert{2}
      \alert{6}
      \alert{9}
      \alert{0}
       Los datos ordenados son:
      0
      1
      2
      3
      4
      5
      6
      7
      8
      9
    \end{Verbatim}

\end{frame}

  \begin{frame}[fragile]
    Tarea para fiestas patrias:
    ordenar palabras por largo.

    \begin{Verbatim}
       Ingrese 4 palabras:
      \alert{nuez}
      \alert{paralelepípedo}
      \alert{perro}
      \alert{bicicleta}
       Ordenadas por largo:
      paralelepípedo
      bicicleta
      perro
      nuez
    \end{Verbatim}

\end{frame}

  \begin{frame}[fragile]
    \begin{Verbatim}
       Ingrese los nombres de los equipos:
      \alert{España}
      \alert{Suiza}
      \alert{Honduras}
      \alert{Chile}
      
       España-Suiza
      \alert{0 1}
       España-Honduras
      \alert{3 0}
       España-Chile
      \alert{2 1}
       Suiza-Honduras
      \alert{0 0}
       Suiza-Chile
      \alert{0 1}
       Honduras-Chile
      \alert{0 1}
      
       Equipo             Pts         Dif
       España               6           3
       Suiza                4           0
       Honduras             1          -4
       Chile                6           1
    \end{Verbatim}

\end{frame}

  \begin{frame}[fragile]
    Búsqueda en arreglos de múltiples dimensiones.
    \begin{Verbatim}
       Ingrese los datos del arreglo
      \alert{1.5  7.1  8.4  9.0}
      \alert{2.1  4.4  4.1  3.2}
      \alert{0.4  0.5  1.9  5.5}
      \alert{3.4  2.7  1.1  0.0}

       ¿Qué elemento quiere buscar?
      \alert{4.1}
       
       Está en la posición i=2, j=3
    \end{Verbatim}

\end{frame}
\end{document}


