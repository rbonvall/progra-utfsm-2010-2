\documentclass[12pt]{beamer}
\usepackage[spanish]{babel}
\usepackage[utf8]{inputenc}
\usepackage{xcolor}
\usepackage{listings}
\usepackage{palatino}
\usepackage{fancyvrb}
\usepackage{amsmath}
\usepackage{graphics}

\usecolortheme{crane}
\usefonttheme{serif}

\title{Ejercicios de archivos}
\author{
  Roberto Bonvallet \\
  \url{roberto.bonvallet@usm.cl} \\
  \url{http://progra.8o.cl}
}

\lstloadlanguages{fortran}
\lstset{language=fortran,basicstyle=\small,%
        morestring=[b]',stringstyle=\it,showstringspaces=false}
\fvset{formatcom=\small,frame=single,gobble=6,commandchars=\\\{\}}

\begin{document}
  \begin{frame}
    \maketitle
  \end{frame}

  \begin{frame}[fragile]
    \frametitle{Ruteo certamen 3, primer semestre 2010}
    Realice el ruteo del siguiente programa (\ldots),
    considerando que los archivos
    \texttt{alumnos.dat} y \texttt{rangos.txt}
    tienen los siguientes datos:
    \vspace{2ex}

    \begin{columns}[t]
      \column{0.6\textwidth}
      \texttt{alumnos.dat}:
      \begin{Verbatim}
      Carola Oyanedel   25  F
      Pedro Osorio      21  M
      Alfonso Lira      19  M
      Gabriela Veliz    18  F
      Andres Fuentes    17  M
      Paula Salazar     23  F
      \end{Verbatim}
      \column{0.4\textwidth}
      \texttt{rangos.txt}:
      \begin{Verbatim}
      25  34
      56  103
      91  124
      \end{Verbatim}
    \end{columns}
    \vspace{2ex}

    Además, escriba cómo quedará el archivo \texttt{rangos.txt}
    después de la ejecución del programa.

\end{frame}

  \begin{frame}
    \frametitle{Pregunta 3, primer semestre 2010}
    El Ministerio de Salud desea conocer la situación
    del virus AH1N1 a nivel nacional.
    Para ello cuenta con el archivo \texttt{ah1n1.dat}
    con registros de tipo \texttt{influenza} (\ldots).

    El ministerio desea realizar una nueva campaña de vacunación,
    en la que cada hospital beneficiario
    deberá vacunar el doble de lo realizado hasta el momento.
    Sólo serán beneficiados aquellos hospitales
    que tengan más de 10 infectados.

    Escriba un programa que:
    \begin{itemize}
      \item muestre en pantalla el total nacional de infectados
        y vacunados a partir del archivo \texttt{ah1n1.dat},
      \item cree el archivo \texttt{vacunar.dat}
        con registros de tipo \texttt{vacuna} (\ldots)
        que tenga el nombre del hospital,
        el número de vacunas que recibirá
        y la ciudad.
    \end{itemize}

  \end{frame}

\end{document}
