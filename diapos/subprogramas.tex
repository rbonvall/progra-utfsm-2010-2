\documentclass[10pt]{beamer}
\usepackage[spanish]{babel}
\usepackage[utf8]{inputenc}
\usepackage{xcolor}
\usepackage{listings}
\usepackage{palatino}
\usepackage{fancyvrb}
\usepackage{amsmath}

\usecolortheme{crane}
\usefonttheme{serif}

\title{Subprogramas}
\author{
  Roberto Bonvallet \\
  \url{roberto.bonvallet@usm.cl} \\
  \url{http://progra.8o.cl}
}

\lstloadlanguages{fortran}
\lstset{language=fortran,basicstyle=\small,%
        morestring=[b]',stringstyle=\it,showstringspaces=false}
\fvset{formatcom=\small,frame=single,gobble=6,commandchars=\\\{\}}

\begin{document}
  \begin{frame}
    \maketitle
  \end{frame}

  \begin{frame}[fragile]
    \begin{columns}
      \column{.45\textwidth}
        Factorial:
        \[ n! = 1\cdot2\cdot\;\cdots\;\cdot(n - 1)\cdot n \]
      \column{.45\textwidth}
        Número combinatorio:
        \[ \binom{m}{n} = \frac{m!}{(m - n)!\;n!} \]
    \end{columns}

    \begin{columns}
      \pause
      \column{.45\textwidth}
        \begin{lstlisting}[gobble=10,frame=single]
          f = 1
          do i = 1, n
              f = f * i
          end do
        \end{lstlisting}

      \pause
      \column{.45\textwidth}
        \begin{lstlisting}[gobble=10,frame=single]
          comb = 1
          f = 1
          do i = 1, m
              f = f * i
          end do
          comb = comb * f
          f = 1
          do i = 1, m - n
              f = f * i
          end do
          comb = comb / f
          f = 1
          do i = 1, n
              f = f * i
          end do
          comb = comb / f
        \end{lstlisting}
    \end{columns}
    
\end{frame}

  \begin{frame}[fragile]
    \[ \frac{\sqrt{m}}{\sqrt{m - n} \sqrt{n}} \]
    \begin{lstlisting}[gobble=6,frame=single]
      c = sqrt(m) / (sqrt(m - n) * sqrt(n))
    \end{lstlisting}

    \[ \frac{\cos{m}}{\cos{(m - n)} \cos{n}} \]
    \begin{lstlisting}[gobble=6,frame=single]
      c = cos(m) / (cos(m - n) * cos(n))
    \end{lstlisting}

    \[ \frac{m!}{(m - n)!\;n!} \] \pause
    \begin{lstlisting}[gobble=6,frame=single]
      c = factorial(m) / (factorial(m - n) * factorial(n))
    \end{lstlisting}
\end{frame}

  \begin{frame}[fragile]
    \begin{lstlisting}[gobble=6,frame=single]
      function factorial(p) result(f)
          integer :: p, f, i
          f = 1
          do i = 1, p
              f = f * i
          end do
      end function factorial
    \end{lstlisting}
    \pause
    \begin{lstlisting}[gobble=6,frame=single]
      function comb(m, n) result(c)
          integer :: m, n, c
          c = factorial(m) / (factorial(m - n) * factorial(n))
      end function comb
    \end{lstlisting}
    
\end{frame}

  \begin{frame}
    \lstinputlisting{../programas/comb-con-funcion.f95}
  \end{frame}

  \begin{frame}[fragile]
    \begin{lstlisting}[gobble=6,frame=single]
      n = factorial(7)
      m = factorial(n)
      p = factorial(1, 2)
      q = factorial('siete')
      s = trim(factorial(10))
      n = factorial(2 * m) + 3
      print *, factorial(n)
      read *, factorial(n)
      factorial(2)
    \end{lstlisting}
\end{frame}

  \begin{frame}[fragile]
    Escriba un programa que diga si el número ingresado
    es primo o no.
    \begin{Verbatim}
       Ingrese un número entero:
      \alert{19}
       19 es primo
    \end{Verbatim}

    Escriba un programa que muestre los 100 primeros números primos.
    \begin{Verbatim}
       2 3 5 7 11 ... 541
    \end{Verbatim}

    Escriba un programa que muestre los números primos menores que 1000.
    \begin{Verbatim}
       2 3 5 7 11 ... 997
    \end{Verbatim}
\end{frame}

\end{document}
