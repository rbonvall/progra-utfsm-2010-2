\documentclass[10pt]{beamer}
\usepackage[spanish]{babel}
\usepackage[utf8]{inputenc}
\usepackage{xcolor}
\usepackage{listings}
\usepackage{palatino}
\usepackage{fancyvrb}

\usecolortheme{crane}
\usefonttheme{serif}

\title{Arreglos}
\author{
  Roberto Bonvallet \\
  \url{roberto.bonvallet@usm.cl} \\
  \url{http://progra.8o.cl}
}

\lstloadlanguages{fortran}
\lstset{language=fortran,basicstyle=\small,%
        morestring=[b]',stringstyle=\it,showstringspaces=false}
\fvset{formatcom=\small,frame=single,gobble=6,commandchars=\\\{\}}

\begin{document}
  \begin{frame}
    \maketitle
  \end{frame}

  \begin{frame}{Promedio}
    \lstinputlisting{../programas/promedio-arreglo-0.f95}
  \end{frame}

  \begin{frame}{Promedio con arreglo}
    \lstinputlisting{../programas/promedio-arreglo-1.f95}
  \end{frame}

  \begin{frame}{Promedio con arreglo con ciclo}
    \lstinputlisting{../programas/promedio-arreglo-2.f95}
  \end{frame}

  \begin{frame}{Tamaño del arreglo como constante}
    \lstinputlisting{../programas/promedio-arreglo-3.f95}
  \end{frame}

  \begin{frame}{Funciones que operan sobre arreglos}
    \lstinputlisting{../programas/promedio-arreglo-4.f95}
  \end{frame}

  \begin{frame}{Inicialización de arreglos}
    \lstinputlisting{../programas/dias-mes.f95}
  \end{frame}

  \begin{frame}[fragile]
    Escriba un programa que pida diez números reales
    y entregue el promedio y la desviación estándar.
    \begin{align}
      \bar x   &= \frac{1}{N} \sum_{i=1}^{N} x_i \\
      \sigma_x &= \sqrt{\frac{1}{N-1} \sum_{i=1}^{N} (x_i - \bar x)^2}
    \end{align}

    \begin{Verbatim}
       Ingrese 10 datos
      \alert{3}
      \alert{5}
      \alert{4.1}
      \alert{2.3}
      \alert{6.7}
      \alert{3.15}
      \alert{-10}
      \alert{4}
      \alert{3}
      \alert{2.1}
       Promedio:   2.3350000    
       Desviacion:   4.5436187
    \end{Verbatim}

\end{frame}

  \begin{frame}{Desviación estándar}
    \lstinputlisting[basicstyle=\tiny]{../programas/desviacion-estandar.f95}
  \end{frame}

  \begin{frame}{Desviación estándar con operaciones sobre arreglos}
    \lstinputlisting{../programas/desviacion-estandar-2.f95}
  \end{frame}

  \begin{frame}[fragile]{Tarea}
    Escriba un programa que pida diez números enteros
    e indique cuál es el que más se repite.

    \begin{Verbatim}
       Ingrese 10 enteros
      \alert{3}
      \alert{5}
      \alert{3}
      \alert{4}
      \alert{1}
      \alert{5}
      \alert{2}
      \alert{5}
      \alert{3}
      \alert{5}
       La moda es 5
    \end{Verbatim}

\end{frame}

\end{document}
