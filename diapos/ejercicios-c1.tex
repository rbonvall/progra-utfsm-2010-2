\documentclass[10pt]{beamer}
\usepackage[spanish]{babel}
\usepackage[utf8]{inputenc}
\usepackage{xcolor}
\usepackage{listings}
\usepackage{palatino}
\usepackage{fancyvrb}

\usefonttheme{serif}

\title{Ejercicios para el certamen 1}
\author{
  Roberto Bonvallet \\
  \url{roberto.bonvallet@usm.cl} \\
  \url{http://progra.8o.cl}
}

\lstloadlanguages{fortran}
\lstset{language=fortran,basicstyle=\small,%
        morestring=[b]',stringstyle=\it,showstringspaces=false}
\fvset{formatcom=\small,frame=single,gobble=6,commandchars=\\\{\}}

\begin{document}
  \begin{frame}
    \maketitle
  \end{frame}

  \begin{frame}[fragile]
    Escriba un programa
    que muestre el promedio de tres números reales
    ingresados por el usuario.

    \begin{Verbatim}
       Ingrese tres numeros:
      \alert{7}
      \alert{1}
      \alert{5.5}
         4.5000000
    \end{Verbatim}

\end{frame}

  \begin{frame}
    \lstinputlisting{../programas/promedio-3.f95}
  \end{frame}

  \begin{frame}[fragile]
    Escriba un programa que:
    \begin{itemize}
      \item pregunte al usuario cuántos números ingresará,
      \item pida al usuario que ingrese los números, y
      \item muestre el promedio de los números.
    \end{itemize}

    \begin{Verbatim}
       Cuantos numeros ingresara?
      \alert{5}
       Ingrese los numeros
      \alert{1.3}
      \alert{2.7}
      \alert{2}
      \alert{3.14}
      \alert{8}
       El promedio es   3.4280000
    \end{Verbatim}

\end{frame}

  \begin{frame}
    \lstinputlisting{../programas/promedio-n.f95}
  \end{frame}

  \begin{frame}[fragile]
    Escriba un programa que:
    \begin{itemize}
      \item pida al usuario que ingrese varios números,
      \item se detenga cuando encuentre un número negativo, y
      \item muestre el promedio de todos los números ingresados.
    \end{itemize}

    \begin{Verbatim}
       Ingrese los numeros
      \alert{1}
      \alert{2}
      \alert{3}
      \alert{4}
      \alert{-7}
       El promedio es   2.5000000
    \end{Verbatim}

\end{frame}

  \begin{frame}
    \lstinputlisting{../programas/promedio-hn.f95}
  \end{frame}

  \begin{frame}
    \lstinputlisting{../programas/promedio-hn-inf.f95}
  \end{frame}

  \begin{frame}[fragile]
    Escriba un programa que pida al usuario que ingrese 10 valores,
    y muestre el menor y el mayor de los números ingresados.

    \begin{Verbatim}
       Ingrese 10 valores
      \alert{67}
      \alert{43}
      \alert{42}
      \alert{15}
      \alert{91.2}
      \alert{4}
      \alert{95}
      \alert{6}
      \alert{44}
      \alert{43}
       El menor es   4.0000000
       El mayor es   95.000000
    \end{Verbatim}

\end{frame}

  \begin{frame}
    \lstinputlisting{../programas/mayor-y-menor.f95}
  \end{frame}

  \begin{frame}
    \lstinputlisting{../programas/mayor-y-menor-b.f95}
  \end{frame}

  \begin{frame}[fragile]
    Escriba un programa que cuente
    cuántos dígitos tiene un número entero
    ingresado por el usuario.

    \begin{Verbatim}
       Ingrese un numero entero
      \alert{142857}
       El numero tiene           6 digitos
    \end{Verbatim}

\end{frame}

  \begin{frame}
    \lstinputlisting{../programas/cuenta-digitos.f95}
  \end{frame}

\end{document}
